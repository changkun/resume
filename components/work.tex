\ifthenelse{\boolean{en}} {
\section{\textbf{Work Experiences}}
}{
\section{\textbf{工作经历}}
}


  \resumeSubHeadingListStartNoLabel

  \ifthenelse{\boolean{en}} {
    \resumeSubheading
      {Research Assistant}{Munich, Germany}
      {University of Munich}{Aprl. 2018 -- Present}
      \resumeItemListStart
      \resumeItem{Instructor}{I am responsible for the design and teaching of the \href{http://www.medien.ifi.lmu.de/lehre/ws2021/gp/}{Practical Geometry Processing} course. In the class, I mainly teach students about geometry processing algorithms, practice how to implement them from scratch, and eventually writing reproducible rendering scripts in Blender. Resources are on \href{https://github.com/mimuc/gp-ws2021}{GitHub}.}
      \resumeItem{DevOps}{I am responsible for compatibility development work and the execution of the eventual migration operation of a 15 years old PHP and SVN-based CMS system that was developed in 2005. Resources are on \href{https://github.com/changkun/destrictor}{GitHub}.}
      \resumeItem{Teaching assistant}{I am responsible for the design and organization of the practical part of the Lecture \href{http://www.medien.ifi.lmu.de/lehre/ss20/cg1/}{Computer Graphics}. Behind the scene, the whole coding exercises are redesigned for the fit of modern topics in graphics. Resources are on \href{https://github.com/mimuc/cg1-ss20}{GitHub}.}
      \resumeItem{Teaching assistant}{I am one of the responsible people for the design and organization of the practical part of Lecture \href{http://www.medien.ifi.lmu.de/lehre/ws1920/omm/}{Online Multimedia}. Behind the scene, I bring novel web development topics into the teaching, such as React, Docker, Kubernetes, etc. Resources are on \href{https://github.com/mimuc/omm-ws1920}{GitHub}.}
      \resumeItem{Teaching assistant}{I am responsible for the organization of \href{http://www.medien.ifi.lmu.de/lehre/ws1920/hs/}{Seminar Advances in Computer Graphics}.}
      \resumeItem{Tutor}{\href{http://www.dbs.ifi.lmu.de/cms/studium_lehre/lehre_master/deep1819/index.html}{Deep Learning and Artificial Intelligence}, notes on \href{https://github.com/changkun/ss18-machine-learning-tutorial}{GitHub}}
      \resumeItem{Tutor}{\href{http://www.dbs.ifi.lmu.de/cms/studium_lehre/lehre_master/ml18/index.html}{Machine Learning}, notes on \href{https://github.com/changkun/ws-18-19-deep-learning-tutorial}{GitHub}.}
      \resumeItemListEnd
  }{
    \resumeSubheading
      {研究助理}{慕尼黑, 德国}
      {慕尼黑大学}{2018.04 -- Present}
      \resumeItemListStart
      \resumeItem{讲师}{我负责了\href{http://www.medien.ifi.lmu.de/lehre/ws2021/gp/}{实用几何处理}课程的设计和教学。在课堂里,我主要教授学生有关几何处理的算法知识,练习如何从头开始实现这些算法,以及
      在 Blender 中编写可复现的渲染脚本。相关教学资源在 \href{https://github.com/mimuc/gp-ws2021}{GitHub} 和 \href{https://www.medien.ifi.lmu.de/lehre/ws2021/gp/}{课程主页}上公开可见。}
      \resumeItem{DevOps}{我负责并主导了针对在 2005 年开发的具有超过 15 年历史的基于 PHP 和 SVN 的 CMS 系统进行的兼容性开发以及最终迁移的运维工作。相关源代码在 \href{https://github.com/changkun/destrictor}{GitHub} 公开可见。}
      \resumeItem{教学助理}{我负责了\href{http://www.medien.ifi.lmu.de/lehre/ss20/cg1/}{计算机图形学} 实践教学部分的设计和组织工作。包括对教学大纲的重新设计、对学生课后编码项目习题的重新设计,进而使内容更加贴合现代图形学最新进展。相关教学资源在 \href{https://github.com/mimuc/cg1-ss20}{GitHub} 和 \href{https://www.medien.ifi.lmu.de/lehre/ss20/cg1/}{课程主页}上公开可见。}
      \resumeItem{教学助理}{我负责了 \href{http://www.medien.ifi.lmu.de/lehre/ws1920/omm/}{在线媒体技术}实践教学部分的设计和组织工作。包括对教学大纲的重新设计、对学生课后编码项目习题的重新设计,进而将更多现代 Web 技术主题引入到教学中,例如 React, Docker, Kubernetes 等等。相关教学资源在 \href{https://github.com/mimuc/omm-ws1920}{GitHub} 和\href{https://www.medien.ifi.lmu.de/lehre/ws1920/omm/}{课程主页}上公开可见。}
      \resumeItem{教学助理}{我负责并组织了\href{http://www.medien.ifi.lmu.de/lehre/ws1920/hs/}{计算机图形学进展} 研讨班。}
      \resumeItem{学生助教}{我负责了\href{http://www.dbs.ifi.lmu.de/cms/studium_lehre/lehre_master/deep1819/index.html}{深度学习和人工智能}课程的实践部分的教学, 教学笔记在 \href{https://github.com/changkun/ss18-machine-learning-tutorial}{GitHub} 上公开可见。}
      \resumeItem{学生助教}{我负责了\href{http://www.dbs.ifi.lmu.de/cms/studium_lehre/lehre_master/ml18/index.html}{机器学习}课程的事件部分的教学,教学笔记在 \href{https://github.com/changkun/ws-18-19-deep-learning-tutorial}{GitHub} 上公开可见}
      \resumeItemListEnd
  }

  \ifthenelse{\boolean{en}} {
    \resumeSubheading
      {Backend Software Engineer (Remote)}{Chengdu, China}
      {\href{https://labex.io/}{LabEx Technology Ltd}}{Apr. 2018 -- Jan. 2019}
      \resumeItemListStart
      \resumeItem{Team leader and leading backend development of the oversea product}{
        I lead and responsible for the product development in backend and frontend. 
        I evolve the existing architecture and split a monolithic backend web application 
        into multiple microservices. 
        The product scales machine cluster from 20 to 200 for active daily users, 
        and its user group increases from 5k to 30k during my incumbency.
      }
      \resumeItem{Remote desktop control proxy}{
        I responsible for and developed a middleware that provides generic remote desktop proxy in Go. 
        The proxy translates VNC/RDP/SSH protocol data, 
        and establish WebSocket connection to a web browser for providing remote desktop GUI.
      }
      \resumeItem{Multi-cloud automation}{
        I developed a fully automated multi-cloud resource management microservice in Go. 
        The service defines a general abstraction cross all cloud provider, 
        it automatically manages all user requested resources allocation and 
        releases outdated resources. 
        For instance, a user of the service can allocate new cloud instances 
        for temporal using without noticing the instance was allocated in either AWS, AlibabaCloud, or others. 
        The service supports more than 15 cloud products and integrated 3 cloud providers, 
        being able to support almost unlimited concurrent users and has been used by 10k+ users.
      }
      \resumeItem{Cluster management service}{
        I developed a microservice in Go that similar to Kubernetes and Docker Swarm. 
        The service manages multiple server clusters, 
        and auto-scaling its cluster size upon request cross multiple cloud providers. 
        Each cluster contains multiple physical machines, 
        and each machine runs many docker containers. 
        The key feature of the service eliminates the difference between the physical machine 
        and the docker container. 
        The runtime of the service includes a system monitor with request prediction algorithm 
        that I invented for efficient auto-scaling with consideration of overcommit ratio 
        and a task scheduler for managing all distributed asynchronous task execution 
        with two-level caching optimization.
      }
      \resumeItem{Used tech. stack}{
        Vue, jQuery, Webpack, Electron; 
        Backend: Go, Cgo, Gin, Beego, gRPC, MySQL, MongoDB, Redis, Hypervisor, Nginx, 
        Docker, Kubernetes, AWS, AlibabaCloud, etc.}
      \resumeItemListEnd
  }{
    \resumeSubheading
    {后端工程师 (远程)}{成都, 中国}
    {\href{https://labex.io/}{LabEx Technology Ltd}}{2018.04 -- 2019.01}
    \resumeItemListStart
    \resumeItem{海外产品的团队和后端开发负责人}{
      我领导并负责后端和前端的产品开发。主导了产品架构的改进,并将一个整体的后端 Web
      应用程序拆分为多个独立的微服务。对于日常活跃用户,让产品支持了将机器集群范围从
      20 个扩展到 200 多个。在任职期间,其用户组从 5k 增加到了 30k。
    }
    \resumeItem{远程桌面控制代理}{
      我负责并开发了一种中间件,该中间件使用 Go 编写,并提供了通用的远程桌面代理服务。
      该代理会转换 VNC/RDP/SSH 协议数据到一种定制的协议,并建立与 Web 浏览器的
      WebSocket 连接,进而提供远程桌面 GUI 支持。
    }
    \resumeItem{多云自动化服务}{
      我设计并开发了一种全自动的多云资源管理微服务。该服务定义了一种建立在所有公有云服务商服务的通用抽象,
      它会自动管理所有用户请求的资源分配,并且自动释放过时的资源。例如,
      服务的用户可以分配新的云实例来临时使用,而不会注意到实例是在 AWS、AlibabaCloud
      或其他云服务商中分配的。该服务抽象支持超过 15 种云产品。在职期间共集成了 3 种云提供商,
      并已被超过1万名用户使用。
    }
    \resumeItem{集群管理服务}{
      我设计并开发了一种与 Kubernetes 和 Docker Swarm 类似但高度契合业务场景的微服务。
      该服务管理多个服务器群集,并根据要求跨多个云提供商自动扩展其群集大小。每个群集包含多个物理机,
      每台机器都运行许多 Docker 容器。服务的关键功能消除了物理机器和 Docker 容器对用户层的差异。
      服务的运行时包括带有请求预测算法的系统监视器。我发明了一种有效的集群自动伸缩算法,
      并考虑了过量使用率和一个任务计划程序,用于管理所有分布式异步任务执行,具有两级缓存优化。
    }
    \resumeItem{涉及的技术栈}{
      Backend: Go, Cgo, Gin, Beego, gRPC, MySQL, MongoDB, Redis, Hypervisor, Nginx, 
      Docker, Kubernetes, AWS, AlibabaCloud, etc;
      Frontend: Vue, jQuery, Webpack, Electron. }
    \resumeItemListEnd
  }

  \ifthenelse{\boolean{en}} {
    \resumeSubheading
    {Fullstack Engineer (Freelance)}{Munich, Germany}
    {\href{https://magiclingua.com/}{Rocketlingo UG}}{Nov. 2017 -- Mar. 2018}
    \resumeItemListStart
      \resumeItem{Language teaching voice bot}
        {I am part of the team in developing a voice bot that provides English learning teaching service. 
        The bot can communicate with its user and improve their English skill by the real-time response. 
        My responsibility is to implement the backend support designed conversations using Amazon Alexa.}
      \resumeItem{Speech recognition solution \& Web development}{
        I responsible for the development of speech recognition solution over web technologies, 
        such as using WebSocket for audio streaming, 
        using Google Cloud STT and TTS services for speech recognition and synthesis, etc. 
        The challenging part of using existing speech recognition service for 
        a language learning application is a new language learner sometimes does not produces 
        positive audio samples, and even multilingual. 
        Therefore, I developed many text-based falt tolerances technique 
        for improving the understanding of user speech based on machine learning algorithms.
      }
    \resumeItemListEnd
  }{
    \resumeSubheading
    {全栈工程师 (兼职)}{慕尼黑, 德国}
    {\href{https://magiclingua.com/}{Rocketlingo UG}}{2017.11 -- 2018.03}
    \resumeItemListStart
      \resumeItem{语言教学语音机器人}
        {我是开发语音机器人的团队的一员,该机器人提供英语学习教学服务。机器人可以通过
        实时响应与用户进行交流并提高他们的英语水平。我的责任是使用 Amazon Alexa 实施
        后端支持设计的对话。}
      \resumeItem{语音识别解决方案及 Web 端支持}{
        我负责基于网络技术的语音识别解决方案的开发,例如使用 WebSocket 进行音频流传输,
        使用 Google Cloud STT 和 TTS 服务进行语音识别和合成等。
        将现有语音识别服务用于以下方面的挑战性部分:语言学习应用程序是一种新的语言学习器,
        有时不会产生正面的音频样本,甚至是多语种。因此,我开发了许多基于文本的容错技术
        用于基于机器学习算法提高对用户语音的理解。
      }
    \resumeItemListEnd
  }

  \ifthenelse{\boolean{cv}}{

    \ifthenelse{\boolean{en}} {
      \resumeSubheading
        {\href{https://shiyanlou.com/}{Shiyanlou}}{Chengdu, China}
        {Software Engineer (Intern)}{Jun. 2016 -- Sep. 2016}
        \resumeItemListStart
          \resumeItem{Recommendation System}
            {Online course recommendation system; Python; MongoDB; Collaborate Filtering}
          \resumeItem{Distributed Log System}
            {Elasticsearch; Logstash; Kibana; Redis}
          \resumeItem{Cross Platform Desktop Client}
            {Electron based development for macOS/Windows/Linux}
          \resumeItem{Teaching Meterial Writing}
            {Modern C++ based teaching meterial writing}
        \resumeItemListEnd

      \resumeSubheading
        {Academic Tutor}{Chengdu, China}
        {SWUN Lecture ``Human-Computer Interaction''}{Jul. 2015}
    }{
      \resumeSubheading
      {\href{https://shiyanlou.com/}{实验楼}}{成都, 中国}
      {软件工程师 (实习)}{2016.06 -- 2016.09}
      \resumeItemListStart
        \resumeItem{推荐系统}
          {负责并开发了基于协同过滤的课程的推荐系统;使用 Python; MongoDB}
        \resumeItem{后端日志收集}
          {负责并搭建了基于 Elasticsearch/Logstash/Kibana/Redis 的日志搜集系统}
        \resumeItem{跨平台桌面客户端}
          {负责并开发了基于 Electron 的跨平台客户端}
        \resumeItem{技术写作}
          {负责编写了多项关于 C++ 的教学材料}
      \resumeItemListEnd

      \resumeSubheading
        {学生助教}{成都, 中国}
        {``人机交互''讲座的学生助教工作}{Jul. 2015}
    }

  }{}

  \resumeSubHeadingListEnd