\ifthenelse{\boolean{en}} {
\section{\textbf{Professional Experience}}
}{
\section{\textbf{工作经历}}
}
\small
\resumeSubHeadingListStartNoLabel
  \ifthenelse{\boolean{en}} {
    \resumeSubheading
      {Research Associate}{Aprl. 2018 -- Present}
      {\href{https://lmu.de/}{LMU Munich}}{Munich, Germany}
      \resumeItemListStart
      \resumeItem{As researcher}{research on human-in-the-loop machine learning 3D graphics systems}
      \resumeItem{As teacher}{teach 12 classes (as lecturer/instructor/assistant/tutor), and supervising 20+ student theses and seminars}
      \resumeItem{As developer}{develop a 3D mesh processing backend system in collaboration with industry partner \href{https://www.way-ds.de/}{WAY digital solutions}; develop, migrate, and maintain a 17 year-old \href{https://www.medien.ifi.lmu.de/}{university} \href{https://github.com/changkun/destrictor}{CMS system}, and a 13 year-old \href{https://barkeeper.medien.ifi.lmu.de/}{collaborative system}
      }
      \resumeItemListEnd
  }{
    \resumeSubheading
      {博士研究员}{慕尼黑, 德国}
      {慕尼黑大学}{2018.04 -- Present}
      \resumeItemListStart
      \resumeItem{讲师}{我负责了\href{http://www.medien.ifi.lmu.de/lehre/ws2021/gp/}{实用几何处理}课程的设计和教学。在课堂里,我主要教授学生有关几何处理的算法知识,练习如何从头开始实现这些算法,以及
      在 Blender 中编写可复现的渲染脚本。相关教学资源在 \href{https://github.com/mimuc/gp-ws2021}{GitHub} 和 \href{https://www.medien.ifi.lmu.de/lehre/ws2021/gp/}{课程主页}上公开可见。}
      \resumeItem{DevOps}{我负责并主导了针对在 2005 年开发的具有超过 15 年历史的基于 PHP 和 SVN 的 CMS 系统进行的兼容性开发以及最终迁移的运维工作。相关源代码在 \href{https://github.com/changkun/destrictor}{GitHub} 公开可见。}
      \resumeItem{教学助理}{我负责了\href{http://www.medien.ifi.lmu.de/lehre/ss20/cg1/}{计算机图形学} 实践教学部分的设计和组织工作。包括对教学大纲的重新设计、对学生课后编码项目习题的重新设计,进而使内容更加贴合现代图形学最新进展。相关教学资源在 \href{https://github.com/mimuc/cg1-ss20}{GitHub} 和 \href{https://www.medien.ifi.lmu.de/lehre/ss20/cg1/}{课程主页}上公开可见。}
      \resumeItem{教学助理}{我负责了 \href{http://www.medien.ifi.lmu.de/lehre/ws1920/omm/}{在线媒体技术}实践教学部分的设计和组织工作。包括对教学大纲的重新设计、对学生课后编码项目习题的重新设计,进而将更多现代 Web 技术主题引入到教学中,例如 React, Docker, Kubernetes 等等。相关教学资源在 \href{https://github.com/mimuc/omm-ws1920}{GitHub} 和\href{https://www.medien.ifi.lmu.de/lehre/ws1920/omm/}{课程主页}上公开可见。}
      \resumeItem{教学助理}{我负责并组织了\href{http://www.medien.ifi.lmu.de/lehre/ws1920/hs/}{计算机图形学进展} 研讨班。}
      \resumeItem{学生助教}{我负责了\href{http://www.dbs.ifi.lmu.de/cms/studium_lehre/lehre_master/deep1819/index.html}{深度学习和人工智能}课程的实践部分的教学, 教学笔记在 \href{https://github.com/changkun/ss18-machine-learning-tutorial}{GitHub} 上公开可见。}
      \resumeItem{学生助教}{我负责了\href{http://www.dbs.ifi.lmu.de/cms/studium_lehre/lehre_master/ml18/index.html}{机器学习}课程的事件部分的教学,教学笔记在 \href{https://github.com/changkun/ws-18-19-deep-learning-tutorial}{GitHub} 上公开可见}
      \resumeItemListEnd
  }
  \ifthenelse{\boolean{en}} {
    \resumeSubheading
      {Backend Software Engineer (Remote)}{Apr. 2018 -- Jan. 2019}
      {\href{https://labex.io/}{LabEx Technology Ltd}}{Munich, Germany}
      \resumeItemListStart
      \resumeItem{As team leader}{leading developments of an oversea product;
        established microservice based backend architecture;
        the product autoscales cloud instances (on AWS/AlibabaCloud) ranging from 20 to 200;
        the product user group grows from 5k+ to 30k+ during my incumbency.
      }
      \resumeItem{As developer}{1) developed a scalable remote desktop proxy (support WebSocket to VNC/RDP/SSH protocols) using Go; 2) developed an automated multi-cloud resource management microservice that abstracts cross cloud providers (supports AWS/AlibabaCloud over 15 cloud products, e.g., IAM/EC2/VPC/etc), scales and used by 10k+ users; 3) developed a kubernetes-like container and instance hybrid management service.
      }
      \resumeItem{Involved techniques}{
        Frontend: Vue, jQuery, Webpack, Electron;
        Backend: Go, Cgo, Gin, Beego, gRPC, MySQL, MongoDB, Redis, Hypervisor, Nginx,
        Docker, Kubernetes, AWS, AlibabaCloud, etc}
      \resumeItemListEnd
  }{
    \resumeSubheading
    {后端工程师 (远程)}{成都, 中国}
    {\href{https://labex.io/}{LabEx Technology Ltd}}{2018.04 -- 2019.01}
    \resumeItemListStart
    \resumeItem{海外产品的团队和后端开发负责人}{
      我领导并负责后端和前端的产品开发。主导了产品架构的改进,并将一个整体的后端 Web
      应用程序拆分为多个独立的微服务。对于日常活跃用户,让产品支持了将机器集群范围从
      20 个扩展到 200 多个。在任职期间,其用户组从 5k 增加到了 30k。
    }
    \resumeItem{远程桌面控制代理}{
      我负责并开发了一种中间件,该中间件使用 Go 编写,并提供了通用的远程桌面代理服务。
      该代理会转换 VNC/RDP/SSH 协议数据到一种定制的协议,并建立与 Web 浏览器的
      WebSocket 连接,进而提供远程桌面 GUI 支持。
    }
    \resumeItem{多云自动化服务}{
      我设计并开发了一种全自动的多云资源管理微服务。该服务定义了一种建立在所有公有云服务商服务的通用抽象,
      它会自动管理所有用户请求的资源分配,并且自动释放过时的资源。例如,
      服务的用户可以分配新的云实例来临时使用,而不会注意到实例是在 AWS、AlibabaCloud
      或其他云服务商中分配的。该服务抽象支持超过 15 种云产品。在职期间共集成了 3 种云提供商,
      并已被超过1万名用户使用。
    }
    \resumeItem{集群管理服务}{
      我设计并开发了一种与 Kubernetes 和 Docker Swarm 类似但高度契合业务场景的微服务。
      该服务管理多个服务器群集,并根据要求跨多个云提供商自动扩展其群集大小。每个群集包含多个物理机,
      每台机器都运行许多 Docker 容器。服务的关键功能消除了物理机器和 Docker 容器对用户层的差异。
      服务的运行时包括带有请求预测算法的系统监视器。我发明了一种有效的集群自动伸缩算法,
      并考虑了过量使用率和一个任务计划程序,用于管理所有分布式异步任务执行,具有两级缓存优化。
    }
    \resumeItem{涉及的技术栈}{
      Backend: Go, Cgo, Gin, Beego, gRPC, MySQL, MongoDB, Redis, Hypervisor, Nginx,
      Docker, Kubernetes, AWS, AlibabaCloud, etc;
      Frontend: Vue, jQuery, Webpack, Electron}
    \resumeItemListEnd
  }
  \ifthenelse{\boolean{en}} {
    \resumeSubheading
    {Fullstack Engineer (Freelance)}{Nov. 2017 -- Mar. 2018}
    {\href{https://magiclingua.com/}{Rocketlingo UG}}{Munich, Germany}
    \resumeItemListStart
    \resumeItem{As developer}{developed a voice bot to support novice language learner to improve their language skills by real-time voice recognition and synthesis (supports web and Amazon Alexa), optimize for audio streaming and multilingual falt tolerances using machine learning
    }
    \resumeItem{Involved techniques}{
      TypeScript; WebSocket; Angular; Google Cloud STT and TTS; Sklearn; Voice Recognition; etc}
    \resumeItemListEnd
  }{
    \resumeSubheading
    {全栈工程师 (兼职)}{慕尼黑, 德国}
    {\href{https://magiclingua.com/}{Rocketlingo UG}}{2017.11 -- 2018.03}
    \resumeItemListStart
      \resumeItem{语言教学语音机器人}
        {我是开发语音机器人的团队的一员,该机器人提供英语学习教学服务。机器人可以通过
        实时响应与用户进行交流并提高他们的英语水平。我的责任是使用 Amazon Alexa 实施
        后端支持设计的对话。}
      \resumeItem{语音识别解决方案及 Web 端支持}{
        我负责基于网络技术的语音识别解决方案的开发,例如使用 WebSocket 进行音频流传输,
        使用 Google Cloud STT 和 TTS 服务进行语音识别和合成等。
        将现有语音识别服务用于以下方面的挑战性部分:语言学习应用程序是一种新的语言学习器,
        有时不会产生正面的音频样本,甚至是多语种。因此,我开发了许多基于文本的容错技术
        用于基于机器学习算法提高对用户语音的理解。
      }
    \resumeItemListEnd
  }
  \ifthenelse{\boolean{cv}}{
    \ifthenelse{\boolean{en}} {
      \resumeSubheading
        {Software Engineer (Intern)}{Jun. 2016 -- Sep. 2016}
        {\href{https://shiyanlou.com/}{Shiyanlou}}{Chengdu, China}
        \resumeItemListStart
        \resumeItem{As developer}{developed a cross-platform desktop software using Electron; developed a recommendation system for E-learning recommendation;
        deployed and operates a logging system for internal data analysis
        }
        \resumeItem{As writer}{wrote more than 20+ professional educational materials regarding C++}
        \resumeItem{Involved techniques}{
          C++; Python; MongoDB; Collaborate Filtering; Elasticsearch; Logstash; Kibana; Redis; Electron, etc.}
        \resumeItemListEnd
    }{
      \resumeSubheading
      {\href{https://shiyanlou.com/}{实验楼}}{成都, 中国}
      {软件工程师 (实习)}{2016.06 -- 2016.09}
      \resumeItemListStart
        \resumeItem{推荐系统}
          {负责并开发了基于协同过滤的课程的推荐系统;使用 Python; MongoDB}
        \resumeItem{后端日志收集}
          {负责并搭建了基于 Elasticsearch/Logstash/Kibana/Redis 的日志搜集系统}
        \resumeItem{跨平台桌面客户端}
          {负责并开发了基于 Electron 的跨平台客户端}
        \resumeItem{技术写作}
          {负责编写了多项关于 C++ 的教学材料}
      \resumeItemListEnd
    }
  }{}
\resumeSubHeadingListEnd