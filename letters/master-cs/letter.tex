%% start of file `template.tex'.
%% Copyright 2006-2013 Xavier Danaux (xdanaux@gmail.com).
%
% This work may be distributed and/or modified under the
% conditions of the LaTeX Project Public License version 1.3c,
% available at http://www.latex-project.org/lppl/.


\documentclass[11pt,a4paper,sans]{moderncv}        % possible options include font size ('10pt', '11pt' and '12pt'), paper size ('a4paper', 'letterpaper', 'a5paper', 'legalpaper', 'executivepaper' and 'landscape') and font family ('sans' and 'roman')

% moderncv themes
\moderncvstyle{classic}                            % style options are 'casual' (default), 'classic', 'oldstyle' and 'banking'
\moderncvcolor{green}                              % color options 'blue' (default), 'orange', 'green', 'red', 'purple', 'grey' and 'black'
%\renewcommand{\familydefault}{\sfdefault}         % to set the default font; use '\sfdefault' for the default sans serif font, '\rmdefault' for the default roman one, or any tex font name
%\nopagenumbers{}                                  % uncomment to suppress automatic page numbering for CVs longer than one page

% character encoding
\usepackage[utf8]{inputenc}                       % if you are not using xelatex ou lualatex, replace by the encoding you are using
%\usepackage{CJKutf8}                              % if you need to use CJK to typeset your resume in Chinese, Japanese or Korean

% adjust the page margins
\usepackage[scale=0.75]{geometry}
% \setlength{\hintscolumnwidth}{3cm}                % if you want to change the width of the column with the dates
% \setlength{\makecvtitlenamewidth}{10cm}           % for the 'classic' style, if you want to force the width allocated to your name and avoid line breaks. be careful though, the length is normally calculated to avoid any overlap with your personal info; use this at your own typographical risks...

\geometry{
% %  a4paper,
%  left=40mm,
%  right=40mm,
    top=20mm,
    bottom=20mm
}

% personal data
\name{Ou}{Changkun}
\title{Motivation Application}
\address{Munich, Germany}
\phone{+49~157~7214~2480}
\email{hi@changkun.us}
\homepage{https://changkun.de}
\begin{document}
%-----       letter       ---------------------------------------------------------
% recipient data
\recipient{\large Personal Statement}{
    % Company, Inc.\\
    % 123 somestreet\\
    % some city
}
\date{\today}
\opening{Dear respected Sir or Madam,}
\closing{Yours sincerely,}
\makelettertitle

%In this essay, you should explain the interest and the abilities for studying in the master's program with a detailed description of the previous achievements in the first degree.

It is my great honor that you are taking your time to consider my particular application. 
I am a Human-Computer Interaction (HCI) graduate student who is studying in the group of 
media informatics, Institute of computer science (IFI) at LMU. After a year of studying, 
I archived a total of 78 ECTS in the past two semesters, with a current grade of 2.24. 
However, when I look back over to my learning process, I found that HCI did not fully 
satisfy my desire for computer science related knowledge. After my careful consideration, 
I decided to apply for the Master program of computer science, and I would like to perform 
them in parallel. This personal statement explains how I have my ability to execute them 
simultaneously.

\textbf{Study and research interests}

During my undergraduate education, I learned and practiced a lot of mathematics and 
computer science related knowledge. I believe that computing services should be 
interactively designed and subtly created for consumers, so I applied LMU Master program 
of HCI so that I can understand more knowledge from the human side and use their insights 
to help me improve my computer science skills, such as humanities and product design. 
With the increase of learning and discovering in this area, I realized that an individual 
degree HCI is not enough to create a revolutionary product, it is actively supported by 
computer science (especially when creating a new interaction technique), and I need more 
backgrounds and technology skill accumulations from the side of computer science. To have 
two different degrees in the fields of science is very common, for instance, Prof. 
Ohlbach hold two degrees in mathematics and physics.

% twice effort
Despite the major focus of HCI relates to human rather than machines, I still doubled the 
effort of study in computer science, coding in the most of my time in the field of machine 
intelligence. Recently I am mostly studying in statistic learning theory, the 
generalization of deep models, and engaged in many practical projects related to depth 
models, such as speech recognition and voice assistants for instance. Computer science and 
its unbreakable mathematical logic have become a ``mental seal'' in my mind. The role of 
HCI helps me understand more reasonable machine system, for example, the lecture 
``information visualization'' discusses properties of human visual perception, it 
significantly influenced my cognition in convolutional neural networks.

\textbf{Former achievements and study abilities}

In my undergraduate studies, I received a large number of awards (see C.V.), including 
Meritorious Winner of the American Mathematical Contest In Modeling. Furthermore, I have 
two published journal papers, and finished my Chinese bachelor degree as the best student 
out of 154 speaks for my academic quality. My undergraduate thesis 
``Designing AlternativeContact-free Control Modalities for Smart Watches'' won an award of 
excellent bachelor paper from the college.
For my graduate studies, if you take a close look at my transcript during the first year 
at LMU, you will find that there exist six lectures that are pure computer science lecture, 
and twelve overlapped courses between two programs.

I have already completed 78 ECTS credits, and 72 ECTS credits have been counted under the 
Master HCI Program in the first two semesters of lesson list. A different 72 ECTS applies 
to the program of computer science (According to Dr. Letz). 
In this semester (my third semester of master study), I participate in two different 
seminars, ``Advanced Topics in Mediainformatics'' and ``Advanced Seminar Deep Learning''. 
They can even apply to each of these two different programs, in the meantime, I am still 
attending a lecture ``Deep Learning'' from the department of statistics. Even though, 
there are still many many interesting courses I want to study but can not attend due to my 
time constraints. For instance, ``Introduction to Quantum Computing'' and 
``Complexity theory'' in a winter semester, ``Managing Massively Multiplayer Online Games'', 
``Modern Computer Architecture'' in a summer semester and many more. These courses are 
enough to cover the remains ECTS I needed.

I believe the above achievements are sufficient to illustrate my learning ability. 
Although the Master's grades are somewhat lower than my Bachelor, there are some simple 
factors that I will elaborate in the next section.

\textbf{Concerned issues}

You might consider that my current overall grades are not excellent and competitive with 
the entire LMU computer institute. Just similar to a typical international student, a 
Chinese who was both a non-native German English speaker, together with the challenge of 
moving to Munich, these makes up for the fact that my grades at LMU are not perfect. These 
are the reason why an English course in my first semester (big data) did not perform well, 
but other English courses were all above 2.3, five above 1.7 and three courses have 1.0.

Undoubtedly, German is the most critical problem throughout the entire program process, 
because the current computer science program requires proficiency language skill of German. 
I only met the proficiency skill in English, which is needed by HCI program. Nevertheless, 
this critical issue can be easily solved. In the past one year, while studying HCI, I also 
take my time to learn German with surpassed A1.1 level, and now stay in an A1.2 learning 
phase. I can understand some simple German conversations and read a small paragraph of 
German texts. A few of evidence appears on my transcript, consider the lectures taught in 
pure German (``Knowledge Discovering in Databases II'' conducted English exam, but it held 
in German. In the meantime, ``Design Workshop I'' was held in entirely German, though I 
did not achieve the ideal results, I still passed both of them). Fortunately, LMU IFI is 
opening more and offering more courses in English, and I think German is not an essential 
problem at all for me with enough English proficiency.

\vspace{5mm}

In summary, My ultimate ideal is to become a qualified computer scientist since I was a 
primary school child. I am passionate regarding science and technologies. After a detailed 
analysis, with my face-to-face discussion with Prof. Butz and Dr. Letz, I hold my view, 
and I firmly believe that it is necessary to have two degrees in this field, and I have 
enough ability to complete them at LMU. Please do not hesitate to contact me if you 
require more documents or wish to take an interview, thank you in advance for your reading 
and consideration, and I trust that you will accept my application.

\makeletterclosing

\end{document}
